%-----------------------------------------------------------------------------
% CONCLUSION
%-----------------------------------------------------------------------------
\section{Conclusions}
\label{sec:conclusions}
\color{black}
In this work, a method called Ordinal Mixed-Effects Random Forest (OMERF) is introduced. 
This approach innovatively extends the application of random forest to the analysis of hierarchical data, specifically for ordinal response variables.
The OMERF model replaces the linear combination of fixed-effect covariates in a cumulative linear mixed model with a random forest.
This novel approach makes a valuable contribution to the statistical literature on aggregated models combining mixed-effects models and tree-based methods.
It inherits the flexibility and predictive power of random forests while preserving the structure of mixed-effects models for ordinal response.

To show the performance of the proposed model simulation study is performed. Subsequently, OMERF algorithm is applied to a real world dataset containing  all the personal information stored in the school's electronic register of a prestigious high school in Milan, in order to identify students who may be at risk of failure and generally of predicting their academic progress.

In the simulation study, the performance of the proposed mixed-effects tree-based method was a clear improvement compared to CLM and CLMM models when the data generating process features nonlinear patterns.
In addition, the predictive power of the OMERF model was comparable to the one of the ordinal random forest model.
Overall, the main advantages of the OMERF algorithm are its flexibility and data inspection capability.
By relaxing the linear assumption of the fixed-effects part, the method could model more complex functional forms, detecting also potential interactions
among covariates, for both categorical and continuous variables.

For this reason, the decision to use OMERF over CLMM depends on the complexity of the fixed effects structure, which is often not known in advance.
If fixed-effect covariates exhibit a linear association with the response and are not correlated, CLMM is expected to perform better.
Conversely, if the covariate set is large, and the covariates potentially interact, creating complex patterns in their association with the response, OMERF is expected to outperform parametric methods with predefined functional forms.
At the same time, when data present a hierarchy, the method is able to take into account the dependence structure within observations and to model it.
In the educational data mining context, this aspect is essential in order to better understand students and the settings in which they learn.

In conclusion, OMERF proves to be a powerful and an easily interpretable method, that can deal with a grouped data structure and can be applied to various complex real data challenges.

In the case study, we give a contribution to the learning analytics area. OMERF method, when applied to educational data, can be a useful tool to support the definition of best practices and
new tutoring programmes aimed at enhancing student performances and helping students at risk. 


However, a potential limitation of the OMERF algorithm stems from the initial step where, during the first initialization, the method estimates the target values (\(\eta_{ijc}\)) through a CLM.
This model imposes a linear effect of covariates on a transformation of the response thereby potentially losing the intricacies that the random forest could capture in the subsequent step of the algorithm.
Therefore, a hint for further developments might be to consider using a different model that can successfully capture nonlinear relationships and interactions between covariates and help initialize the \(\eta_{ijc}\) estimates, thereby allowing the OMERF method to perform at its best.

