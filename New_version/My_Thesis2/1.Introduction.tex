%-----------------------------------------------------------------------------
% INTRODUCTION
%-----------------------------------------------------------------------------
\section{Introduction}
\label{sec:introduction}
Ordinal-scale observations are data that can be categorized into a finite and ordered set of discrete categories. However, the distances between these categories can be uneven or unknown.
\textit{"In all fields, ordinal scales result when inherently continuous variables are measured or summarized by researchers by collapsing the possible values into a set of categories. [\dots]
An ordinal variable is quantitative, however, in the sense that each level on its scale refers to a greater or smaller magnitude of a certain characteristic than another level.
Such variables are of quite a different nature than qualitative variables, which are measured on a nominal scale and have categories that do not relate to different magnitudes of a characteristic"} \cite{agresti2010analysis}.

The growing importance of ordinal categorical data shows a clear trend, driven by the increasing use of surveys and tests.
As data collection expands into various areas, there is a bigger need to model ordered data.
This type of data is usful to collect detailed information, especially in fields like market research, public opinion analysis, and healthcare, where assessing opinions, preferences, and responses is extremely important.

Ordinal classification, often known as ordinal regression \cite{mccullagh1980regression}, represents a type of multi-class classification where there is an inherent ordering relationship between the classes, but where there is not a meaningful numeric difference between them.
This type of problem occurs frequently in human created scales, which cover many domains from product reviews to medical diagnosis.

Order becomes relevant when the categories take on meanings related to strength of opinion or agreement (as in a Likert-type response) or frequency.
An explanatory example is the case where a response variable takes on four possible values: (1) strongly disagree, (2) disagree, (4) agree, (5) strongly agree. There is a natural order to the response possibilities.

This paper presents an extension of the standard random forest method \cite{breiman2001random} to ordinal data, that while investigating the learning process, takes also into account the underlying nested structure using mixed-effects models \cite{pinheiro2006mixed}.
The proposed method, called Ordinal Mixed-Effects Random Forest (OMERF), fits into the context of aggregated models.
In particular, the algorithm disentangles the estimations of fixed and random effects. It iteratively fits a random forest ignoring the grouped data structure, and a cumulative mixed-effects model based on the resultant random forest structure; a final mixed-effects random forest is reported.

In the following sections the OMERF method is described, providing a detailed description of the estimation procedure. Subsequently, two applications of the algorithm are proposed: a simulation study, comparing its performance to other existing methods, and a case study with a real world dataset.
The dataset concerns 427 students of a prestigious high school in Milan, containing  all the personal information stored in the school's electronic register.
The motivating problem of this study is to predict early and accurately the student performance, taking into account as ordinal response the final grades of the students in mathematics.
This dataset has been primarily presented in \cite{tesilidia}, nevertheless the analyses carried out in the thesis did not consider the order in the response variable and the nested data structure. Thus, we propose a more flexible and ad hoc method for the problem.

In the last few decades, learning analytics is receiving particular attention.
The primary focus lies in identifying the most effective model in terms of detecting student at risk of failure, in the context of Early Warning Systems.
\textit{"An early warning system is an intentional process whereby school personnel collectively analyze student data to monitor students at risk of falling off track for graduation and to provide the interventions and resources to intervene"} \cite{davis2013organizing}.

There is a growing interest among academics in predicting underperforming young students, driven by the potential benefits of remedial learning that align with the institution's objective of offering high-quality educational environments \cite{wilkins2016dropout}.
Furthermore, the field of data mining in education is rapidly gaining significance as it enables the discovery of valuable insights from a large amount of student data \cite{wibawa2021learning,paul2020exploring}.
A. A. Saa \cite{saa2016educational} defines \textit{"Educational Data Mining (EDM) is a new trend in the data mining and Knowledge Discovery in Databases (KDD) field
which focuses in mining useful patterns and discovering useful knowledge from the educational information systems [\dots]. Researchers in this field
focus on discovering useful knowledge either to help the educational institutes manage their students better, or to help students to manage their education and deliverables better and
enhance their performance"}.


Results reveal that in a nonlinear setting the proposed method performs better or comparably with respect to existing models; providing substantial improvements and having the advantages of increasing the knowledge about the multilevel nature of data and their complex dependencies.

Section \ref{sec:literature_rev} conducts a review of the existing literature related to methods analogous to the one proposed. Section \ref{sec:methods} articulates the OMERF method, outlining its theoretical foundations and its implementation.  Section \ref{sec:sim} is intended
for the description of a simulation study. Section \ref{sec:case} delves into the details of the previously mentioned case study. Ultimately,  Section \ref{sec:conclusions} is dedicated to highlighting conclusions and fostering a discussion.

All the analysis are performed using R software \cite{rlanguage} and all the R codes for the OMERF algorithm and for both simulation and case study are available in the following github repository: \url{https://github.com/giuliabergonzoli/Master_Thesis}.
