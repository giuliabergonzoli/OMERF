\appendix
\section{Appendix A}
\label{sec:appA}

\begin{lstlisting}[language=R]
  omerf= function (y, cov, group, xnam=NULL, znam=NULL, bizero=NULL, 
                 itmax=100, toll=0.05) {
  #ordinal mixed-effects random forest (OMERF)
  #inputs: 
  #-y = vector of responses
  #-cov = data frame with all the fixed covariates for each statistical unit
  #-group = factor vector which, for each statistical unit,
  #         tells the group to which it belongs (random effect)
  #-xnam = vector with the names of the fixed covariates 
  #        to be used in the random forest
  #-znam = vector with the names of the covariates
  #        to be used in the random effects
  #-bizero = matrix containing the coefficients of the random effects
  #          (the first value of each column is the intercept,
  #           the other ones are the covariates znam)
  # assume that group and bizero are coherent, ie b[,i] 
  # corresponds to levels(group)[i]
  
  ###### STEP 1: Initialization  ######

  N <- length(y) #number of observations
  n=length(levels(group)) #number of groups
  
  q <- length(znam)+1	#number of random covariates + random intercept
  Zi=NULL
  z.not.null=!(is.null(znam)) #check whether there are 
                              #covariates included in the random effects
  if (z.not.null) Zi=cov[znam] #random effects covariates
  Zi.int= cbind(rep(1,N),Zi) #random intercept + random effects covariates
  
  #Initialize (if NULL) bi to 0
  if( is.null(bizero) ){
    bi <- NULL
    for(i in 1:n) bi=cbind(bi,rep(0,q))
  }	
  if( !is.null(bizero) ) bi=bizero
  lev=levels(group) #group names
  bi=data.frame(bi)
  names(bi)=lev
  all.bi=list()  #bi of each iteration
  all.bi[[1]]=bi
  
  #if xnam is NULL assume that all cov variables are to be used
  if(is.null(xnam)) xnam=names(cov)
  
  #group must be a factor, otherwise it gives an error
  if(!is.factor(group)) stop("The "group" argument must be a factor")
  
  forest.formula=as.formula(paste("target~", paste(xnam, collapse= "+")))
  
  if(z.not.null) {
    clmm.formula=as.formula(paste("y~offset(f.x_ij)+(1+", 
    paste(znam, collapse= "+"), "|group)"))
  }
  else {clmm.formula=as.formula(paste("y~offset(f.x_ij)+(1|group)"))}
  


  ###### STEP 2: CLM for mu_ij initialization ######

  library(ordinal)
  clm.formula=as.formula(paste("y~", paste(xnam, collapse= "+")))
  clm.data= cbind(y,cov[xnam])
  clm.0=clm(clm.formula, data=clm.data, link="logit")
  mu.ij.0=clm.0$fitted.values #P(yij=c)
  prob.est=predict(clm.0,clm.data,type = "cum.prob")[1]$cprob1
  prob.est[which(prob.est==1)]=0.99999999
  eta.est=qlogis(prob.est) #linear predictor of clm
  
  ###### STEP 3-4: Iterative model estimation  ######

  library(randomForest)
  it=1
  converged=FALSE
  while(!converged && it<itmax) {
    
    #random forest
    target=rep(0,N) #target=eta+Z%*%b
    for (i in 1:N) {
      b.temp=as.matrix(bi[group[i]], nrow=q, ncol=1)
      z.temp=as.matrix(Zi.int[i,], nrow=1, ncol=q)
      target[i]= eta.est[i] + z.temp%*%b.temp
    }
    forest.data=cbind(target, cov[xnam])
    forest=randomForest(forest.formula, forest.data, importance=TRUE) 
    f.x_ij=forest$predicted
    
    clmm.data = data.frame(y,cov[znam],group,f.x_ij)
    
    clmm.fit= clmm(clmm.formula, link="logit", data=clmm.data, Hess=TRUE, 
    control=clmm.control(maxLineIter = 500, maxIter=1000, grtol=1e-3))
    
    #I want to maintain the order of the elements of bi
    select=c("(Intercept)",znam) #all bi to be extracted
    clmm.bi=ranef(clmm.fit)$group[select]
    
    #convergence of bi
    bi.old=bi
    bi=data.frame(t(clmm.bi))
    names(bi)=lev
    diff.t=abs(bi.old-bi)
    n.diff=max(diff.t) #use the infinite norm (max)
    ind=which(diff.t==n.diff, arr.ind=T)
    n.old=abs(bi.old[ind])
    converged= n.diff/n.old <toll
    
    it=it+1
    all.bi[[it]]=bi
  }
 
  ###### STEP 5: Output preparation  ######

  #if there is no convergence it gives an error message
  if(!converged) {
    warning("Maximum number of iterations exceeded, no convergence achieved")
  }
  result=list(clmm.fit,forest,target,bi,it,converged,all.bi,xnam)
  names(result)=c("clmm.model", "forest.model","rf.target",
                  "rand.coef", "n.iteration","converged",
                  "all.rand.coef","forest.var")
  class(result)="omerf"
  result
}


###### Auxiliary functions  ######


summary.omerf=function(om) {
  print("Mixed effects model") #summary of the mixed effects model
  print(summary(om$clmm.model)) 
  str=ifelse(om$converged, "Converged", "Did not converge")
  print(paste(str , "after", om$n.iteration, "iterations"))
  #says whether there is convergence
}


fitted.omerf=function(om, group, type="response") {
  #the correct fitted values are already those of the clmm function,
  #as they incorporate the fitted values of the random forest into the model
  mu_avg=om$clmm.model$fitted.values #for an avg random effect
  f.x_ij=om$forest.model$predicted
  y_train=as.numeric(om$clmm.model$y)
  
  ranef=rowSums(ranef(om$clmm.model)$gr)
  mu=rep(0,length(mu_avg))
  eta=rep(0,length(mu_avg))
  for (j in 1:length(mu_avg)) {
    i=group[j]
    c=y_train[j]-min(y_train)+1
    if(c==1) {
      eta[j]=om$clmm.model$Theta[c] - ranef[i] - f.x_ij[j]
      mu[j]=plogis(eta[j])
    } else if(c==max(y_train)-min(y_train)+1) {
      eta[j]=qlogis(0.99999999)
      mu[j]=1 - plogis(om$clmm.model$Theta[c-1] - ranef[i] - f.x_ij[j])
      # 1 - P(yij<=c-1)
    } else {
      eta[j]=om$clmm.model$Theta[c] - ranef[i] - f.x_ij[j]
      mu[j]=plogis(eta[j]) - 
        plogis(om$clmm.model$Theta[c-1] - ranef[i] - f.x_ij[j])
        # P(yij<=c) - P(yij<=c-1)
    }
  }
  #error if the answer type is not one of the following
  allowed=c("response", "mu", "eta")
  msg=paste("Possible choices are", allowed[1],"," ,allowed[2],"and", 
            allowed[3], sep="")
  if(sum(type==allowed)==0) stop("Type of prediction not available:",msg)
  if(type=="mu") ans=mu # P(yij=c)
  if(type=="response") ans=as.factor(y_train)
  if(type=="eta") ans=eta
  ans
}






predict.omerf=function(om, y_train, newdata, group, type="response", 
                       predict.all=FALSE) {
  forest.data=newdata[om$forest.var]		
  f.x_ij=predict(om$forest.model,forest.data,predict.all=predict.all)
  
  preds <- as.data.frame(matrix(0, dim(newdata)[1], length(levels(y_train))))
  eta <- as.data.frame(matrix(0, dim(newdata)[1], length(levels(y_train))))
  ranef=rowSums(ranef(om$clmm.model)$group)
  lev=levels(y_train)
  names(preds) = lev
  names(eta) = lev
  for (c in 1:length(lev)) {
    for (j in 1:dim(newdata)[1]) {
      i=group[j]
      if(c==1) {
        eta[j,c]=om$clmm.model$Theta[c] - ranef[i] - f.x_ij[j]
        preds[j,c]=plogis(eta[j,c])
      } else if(c==length(lev)) {
        eta[j,c]=qlogis(0.99999999)
        preds[j,c]=1 - plogis(om$clmm.model$Theta[c-1] - ranef[i] - f.x_ij[j])
        # 1 - P(yij<=c-1)
      } else {
        eta[j,c]=om$clmm.model$Theta[c] - ranef[i] - f.x_ij[j]
        preds[j,c]=plogis(eta[j,c]) - 
          plogis(om$clmm.model$Theta[c-1] - ranef[i] - f.x_ij[j])
          # P(yij<=c) - P(yij<=c-1)
      }
    }
  }
  max_mu <- apply(preds, 1, max)
  pred_val <- names(preds)[apply(preds, 1, which.max)]
  
  
  #error if the answer type is not one of the following
  allowed=c("response", "mu", "eta")
  msg=paste("Possible choices are", allowed[1],"," ,allowed[2],"and", 
            allowed[3], sep="")
  if(sum(type==allowed)==0) stop("Type of prediction not available:",msg)
  
  if(type=="eta") ans=eta # eta for the predicted class
  if(type=="mu") ans=preds # P(yij=c) c=1,...,C
  if(type=="response") ans=pred_val
  ans
}


ranef.omerf=function(om, group, znam=NULL) {
  ranef=ranef(om$clmm.model)$gr
  n = (length(znam)+1)*length(levels(group))
  step = length(levels(group))
  par(mfrow=c(1,length(znam)+1))
  for (i in 1:(length(znam)+1)) {
    condVar=rep(0,step)
    if(is.null(znam)) { condVar_temp = om$clmm.model$condVar
    } else { condVar_temp = om$clmm.model$condVar[((i - 1) * step + 1):(i * step),((i - 1) * step + 1):(i * step)] }
    for (l in 1:step) {
      if(is.null(znam)) { 
         condVar[l] <- condVar_temp[l]
      } else {condVar[l] <- condVar_temp[l,l]}
    }
    condVar_sqrt <- sqrt(condVar)
    ci <- ranef[,i] + qnorm(0.975) * condVar_sqrt %o% c(-1, 1)
    ord.re <- order(ranef[,i])
    lev <- levels(group)
    ci <- ci[order(ranef[,i]),]
    plot(1:length(lev), ranef[ord.re,i], axes=FALSE, ylim=range(ci),
         xlab="random variable", ylab="random effect")
    axis(1, at=1:length(lev), labels = lev[ord.re])
    axis(2)
    for(k in 1:length(lev)) segments(k, ci[k,1], k, ci[k, 2])
    abline(h = 0, lty=2)
  }
}


  
\end{lstlisting}